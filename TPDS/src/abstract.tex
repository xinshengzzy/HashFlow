\begin{abstract}
Collecting flow records is a common practice of network operators and researchers 
for monitoring, diagnosing and understanding a network. Traditional tools like 
NetFlow face great challenges when both the speed and the complexity of the network traffic increase. 
To keep pace up, we propose HashFlow, a tool for more efficient and accurate 
collection and analysis of flow records. The central idea of HashFlow is to  
maintain accurate records for elephant flows, but summarized records for mice flows, 
by applying a novel collision resolution and record promotion strategy to hash tables. 
The performance bound can be analyzed with a probabilistic model, 
and with this strategy, HashFlow achieves a better utilization of space, and also 
more accurate flow records, without bringing extra complexity. 
We have implemented HashFlow, as well as several latest flow measurement algorithms such as 
FlowRadar, HashPipe and ElasticSketch, in a P4 software switch. We implement HashFlow in a commodity P4 switch in particular to demonstrate its implementability.
Then we use traces from different operational networks to evaluate the algorithms.
In these experiments, for various types of traffic analysis applications, 
HashFlow consistently demonstrates a clearly better performance against 
its state-of-the-art competitors. 
For example, using a small memory of 1 MB, HashFlow can accurately record 
around 55K flows, which is often 12.5\% higher than the others.
For estimating the sizes of 50K flows, HashFlow achieves a relative error of around 11.6\%, 
while the estimation error of the best competitor is 42.9\% higher. 
It detects 96.1\% of the heavy hitters out of 250K flows with a size estimation error of 5.6\%, which is 11.3\% and 73.7\% better than  
the best competitor respectively.
At last, we show these merits of HashFlow come with almost no degradation of throughput.
\end{abstract}

\section{Control Plane}
\label{section:controlplane}
In the algorithm described above, each flow record occupies up to 17 bytes, including 13 bytes going to the flow ID and another 4 bytes going to the packet count. We argue that this is memory-consuming because the flow ID has to be stored in the precious SRAM to allow real-time access. Moreover, since we have to compare the flow ID of every newly arriving packet with the recorded flow IDs while the flow ID consists of 5 different components, we need at least 3 match-action tables to do comparison (each match-action table can compare 2 components at most). Since the number of match-action tables supported by a P4 switch is extremely limited, this algorithm is deeply flawed. To make full use of the resources supplied by the P4 switch, we proposed to add a control plane to the algorithm, so that we can maintain a 32-byte digest instead of the full flow ID for each flow record, and maintain the full flow records in the control plane. Afterwards, we can recover the flow records using the digests maintained in the data plane and the flow ID maintained in the control plane.

One prime concern about this solution is that several flow IDs may have the same digest, making the recovery impossible. As a rule of thumb, the collision rate is negligible when mapping a flow ID to a 32-bit digest since a 32-bit digest corresponds to a space being able to accommodate 4.3 billion different entries while a 2M-byte memory can accommodate 260K flow records only and we have to export the content of the SRAM whenever there are about 260K flow records there. 

We denote this new version of HashFlow by AHashFlow, i.e., Augmented HashFlow. In AHashFlow, when the switch receives a packet, it will extract its flow ID and map it into a digest. 


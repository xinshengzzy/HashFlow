\documentclass{article}
\title{Experiments Plan}
\author{Zhao Zongyi}
\date{Dec. 22nd, 2018}
\begin{document}
\maketitle
\section*{Exp \#84491}
\begin{itemize}
	\item Use the simulator of HashFlow implemented in python.
	\item Set the memory size to be 1 MB, so it can accommodate around 55K flow records.
	\item Select 10 trace files from each of the four traces.
	\item Initiate 50K flows from each trace file.
	\item Increase the depth of HashFlow from 1 to 4. 
	\item Count the packets processed by the simulator, i.e., the number of original packets plus the resubmitted packets.
\end{itemize}

\section*{Exp \#84492}
\begin{itemize}
	\item Select a file from the CAIDA trace (equinix-nyc.dirA.20180315-125910.UTC.anon.pcap), extract the first 2.5 million packets, classify the TCP/UDP packets into flows, and then calculate the average size as well as the maximum size of the flows.
	\item Select a file from the HGC trace (20080415000.pcap), extract the first 2.5 million packets, classify the TCP/UDP packets into flows, and then calculate the average size as well as the maximum size of the flows.
	\item Calculate the average as well as maximum size of a file from China Telecom trace (nfcapd.201601022000).
	\item Calculate the average as well as maximum size of a file from Tsinghua campus trace (20140206-6). 
\end{itemize}

\section*{Exp \#84493}
\begin{itemize}
	\item Set the memory size to be 1MB. 
	\item Increase the number of flows from 10K to 100K, in the step size of 10K.
	\item Use a trace file from CAIDA and HGC respectively.
	\item Use four versions of HashFlow. In the versions the number of buckets in the ancillary table is $0.25\times$, $0.5\times$, $1.0\times$, and $2\times$ respectively of the number of buckets in main table.
	\item Calculate the average relative error for flow size estimation.
\end{itemize}
\end{document}